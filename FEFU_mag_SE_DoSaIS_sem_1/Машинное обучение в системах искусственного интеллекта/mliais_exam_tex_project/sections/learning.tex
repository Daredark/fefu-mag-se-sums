\section{Learning and etc}

\subsection{2. Содержание}

1. Типология задач обучения по прецедентам.
2. Типы и примеры задач обучения с учителем.
3. Типы и примеры задач обучения без учителя.
4. Другие типы задач обучения.
5. Формальные определения терминов.

\subsection{3. Типология задач обучения по прецедентам}

1. \textbf{Обучение с учителем}
    - Для каждого объекта обучающей выборки известен выход (класс), поэтому
    можно считать, что его указывает некий учитель.
2. \textbf{Обучение без учителя}
    - Для объектов обучающей выборки выходы (классы) не известны.
    Необходимо определить, как объкты связаны друг с другом, например,
    выделить группы (кластеры) близких по своим свойствам объектов.
3. \textbf{Другие типы задач обучения}

\subsection{4. Схема машинного обучения}

\includegraphics[scale=0.3]{figures/samplefigure.jpg}

\subsection{4.1. Далее устная информация}

- Supervised - это обучение с учителем
- Unsupervised - это обучающая выборка
- Test set - контрольная выборка
- Feature extraction - извлечение признаков
- Predictive model - предсказательная модель (классификатор?)
- Annotated data - данные, которые стали классифицированы
- Grouping of objects based on some common characteristics - разбиение
объектов на группы, основываясь на каких-то общих признаках.

\subsection{5. Типы и примеры задач обучения с учителем}

1. \textbf{Задача классификации}
    - Множество допустимых ответов конечно (их называют метками классов).
    Класс - это множество всех объектов с данным значением метки.
2. \textbf{Задача регрессии}
    - Допустимым ответом является действительное число
    (или числовой вектор).
3. \textbf{Задача ранжирования}
    - Ответы надо получить сразу на множестве объектов, после чего
    отсортировать их по значениям ответов.
    Задача может сводиться к задачам классификации или регрессии.
    Часто применяется в информационном поиске и анализе текстов.
4. \textbf{Задача прогнозирования}
    - Объектами являются отрезки временных рядов, обрывающиеся в том момент,
    когда требуется сделать прогноз на будущее.
    Для решения часто удается приспособить методы регрессии или
    классификации.

\subsection{5.1. Далее устная информация}

Релевантность - это степень соответствия найденных данных информационным
нуждам пользователя.

\subsection{6. 1.1 Задача классификации}

\textbf{1.1.1. Задача медицинской диагностики}

\textit{Объекты} - пациенты, \textit{признаки} характеризуют симптомы заболевания,
результаты обследований и применявшиеся методы лечения.

\textit{Бинарные признаки}: пол, "наличие признака".

\textit{Порядковые признаки}: тяжесть состояния.

\textit{Количественные признаки}: возраст, АД, СОЭ.

\textit{История болезни} - совокупность описаний признаков.

\textbf{Подзадачи}: ...

\subsection{7. 1.1 Задача классификации}

\textbf{1.1.2. Задача оценивания заемщиков}

\textit{Объекты} заемщики, \textit{признаки} - анкеты, дополнительная информация из
собственных источников банка.

\textit{Номинальные признаки}: адрес, профессия, работа.

\textit{Бинарные признаки}: пол, наличие телефона.

\textit{Порядковые признаки}: образование, должность.

\textit{Количественные признаки}: возраст, стаж, доход, задолженность, сумма.

\textit{Классификация}: на "хороших" заемщиков и "плохих".

\textbf{Кредитный скоринг} - оценивание суммарного числа баллов, набранных на
совокупности инфопризнаков.

\textbf{Подзадачи}: отбор инфопризнаков и их весов (чтобы риск был минимальным),
определение ставок, срока погашения.

\subsection{8. 1.1 Задача классификации}

\textbf{1.1.3. Задача предсказания ухода клиента}

\textit{Кто решает}: крупные компании, работающие с большим количеством клиентов,
как правило, с физическими лицами (телекоммуникационные компании), для
которых удержание приоритетнее поиска новых клиентов.

\textit{Объекты} - клиенты в моменты времени, \textit{признаки} - анкеты, данные о
частоте пользования услугами компании ...

\subsection{10. 1.2 Задача регрессии}

\textbf{1.2.1. Задача прогнозирования потребительского спроса}

\textit{Кто решает}: супермаркеты и торговые розничные сети.

\textit{Цель}: прогнозировать объемы продаж для каждого товара на заданное число
дней вперед.

\textit{Результат}: на основе прогнозов осуществляется планирование закупок,
управление ассортиментом, формирование ценовой политики и акций.

\textit{Специфика}: количество исчисляется десятками тысяч.

\textit{Обучающая выборка}: данные с кассовых аппаратов.

\textit{Факторы}: реклама, социум, конкуренты, праздники.

\textit{Предмет анализа}: товары, магазины, товары-магазины.

\textit{Особенность}: несимметричность функции потерь (потери от заниженного
прогноза выше, чем от завышенного).

\subsection{12. 1.3 Задача ранжирования}

\textbf{1.3.1. Задача ранжирования текстовых документов}

\textit{Кто решает}: поисковые системы.

\textit{Объекты}: пара "запрос, документ".

\textit{Ответ}: оценки релевантности, сделанные оценщиками (могут быть бинарными
или порядковыми, в баллах).

\textit{Признаки}: числовые характеристики, вычисляемые по паре "запрос, документ".

\textit{Виды признаков}:
- \textbf{текстовые} (подсчет числа вхождений слов запроса в документы: с учетом
синонимов или без, с учетом числа вхождений или без, во всем документе или
только в заголовках и т.д.),
- \textbf{ссылочные} (подсчет числа документов, ссылающихся на данный),
- \textbf{кликовые} (подсчет числа обращений к данному документу).

\subsection{13. 2. Типы и примеры задача обучения без учителя}

\textbf{2.1. Задача кластеризации}

Нужно сгруппировать объекты в класетры, используя данные о попарном сходстве
объектов.
Функционалы качества могут определяться по-разному, например, как
отношение средних межкластерных и внутрикластерных расстояний.

\textbf{2.2. Задача поиска ассоциативных правил}

Исхожные данные представляются в виде признаковых описаний.
Требуется найти такие наборы признаков, и такие значения этих признаков,
которые особенно часто (неслучайно часто) встречаются в признаковых
описаниях объектов.

\textbf{2.3. Задача фильтрации выбросов (детекция аномалий)}

Обнаружение в обучающей выборке небольшого числа нетипичных объектов.
В некоторых приложениях их поиск является самоцелью (например, обнаружение
мошенничества). В других приложениях эти объекты являются следствием ошибок
в данных или неточности модели, то есть шумом, мешающим настраивать модель,
и должны быть удалены из выборки.

\subsection{14. 2. Типы и примеры задача обучения без учителя}

\textbf{2.4. Задача построения доверительной области}

Область минимального объема с достаточно гладкой границей, содержащей
заданную долю выборки. В доверительную область будут попадать наблюдения с
высокой вероятностью, а наблюдения, попавшие за ее пределы, могут быть
отброшены.

\textbf{2.5 Задача сокращения размерности}

Заключается в том, чтобы по исходным признакам с помощью некоторых функций
преобразования перейти к наименьшему числу новых признаков, не потеряв при
это существенной информации об объектах выборки.

\textbf{2.6 Задача заполнения пропущенных значений}

Заключается в замене недостающих значений в матрице объекты-признаки их
прогнозными значениями.

\subsection{16. 2.1. Задачи кластеризации}

\textbf{2.1.1. Задачи рубрикации текстов}

\textit{Обучающая выборка}: множество документов, классифицированных по рубрикатору
вручную.

\textit{Требуется}: классифицировать по тем же рубрикам второе множество
документов, которое может быть существенно больше первого.

\textit{Метод решения}: ...

\subsection{17. 3. Другие типы задач}

\textbf{3.1. Частичное обучение}

Занимает пробежуточное положение между обучением с учителем и без учителя.
Каждый прецедент представляет собой пару "объект, ответ", но ответы
известны только на части прецедентов.
\textit{Пример}: автоматическая рубрикация большого количества текстов при условии,
что часть из них уже отнесена к каким-то рубрикам.

\textbf{3.2. Трансдуктивное обучение}

Обучение с частичным привлечением учетиеля, когда прогноз предполагается
делать только для прецедентов из тестовой выборки ("от чатсного - 
к частному").

\textbf{3.3. Обучение с подкреплением}

Роль объектов играют пары "ситуация, принятое решение", ответами являются
значения функционала качества, характеризующего правильность принятых
решений (реакцию среды).
\textit{Примеры}: формирование инвестиционных стратегий, самообучение роботов,
автоматическое упралвение технологическими процессами и т.д.

\subsection{18. 3. Другие типы задач}

\textbf{3.4 Динамическое обучение}

Может быть как обучением с учителем, так и без учителя.
Специфика в том, что прецеденты поступают потоком.
Требуется немедленно принимать решение по каждому прецеденту и одновременно
доучивать модель зависимости с учетом новых прецедентов.

\textbf{3.5 Активное обучение}

Обучаемый имеет возможность самостоятельно назначать следующий прецедент,
который станет известен.

\textbf{3.6 Метаобучение}

Прецедентами являются ренее решенные задачи обучения.
Требуется определить, какие из используемых в них эвриситк работают наиболее
эффективно.
Конечная цель состоит в обеспечении постоянного автоматического
совершенствования алгоритма с течением времени.